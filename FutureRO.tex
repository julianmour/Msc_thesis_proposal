%! Author = julianmour
%! Date = 01/05/2023

\section{Future Research Objectives}
In light of the preliminary work, we aim to further explore our ideas in the following directions:
\begin{itemize}
    \item Improved algorithm: Maximizing the neighborhood norm, execution time and number of queries -
    The main challenge in using our above-mentioned approach is the calculation of the weakest points that are later used by the optimizer to compute the RL gradient.
    Where in the regular single-label case at most $|C|$ queries are executed per iteration to achieve an accurate and unbiased RL gradient, this number jumps to $|C|^2$ in the multi-label case if the same approach as MaRVeL's is used, leading to a much longer execution time.
    Also, we aim to maximize the neighborhood norm;
    We want to achieve wide robust layer-neighborhoods.
    This can be affected by many factors (e.g.\ the shrinking method used in the algorithm) - we will explore each of these factors and look for the best methods to achieve this goal.
    \item Explainability - The goal of our program is to present a relation between two objects in an image for a specific multi-label classifier.
    The results can tell us how much and where we can change in one object so that it doesn't or does affect the other.
    These can vary between different classifiers as well, which will also help us understand which type of multi-label networks are most vulnerable to perturbations in different locations in their inputs.
\end{itemize}
